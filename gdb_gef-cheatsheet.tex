\documentclass[a4paper,landscape]{article}
\usepackage[T1]{fontenc}
\usepackage{lmodern}
\usepackage[bookmarks=true,colorlinks=true,linkcolor=black,citecolor=black,filecolor=black,urlcolor=black]{hyperref}
\usepackage{multicol}
\usepackage{geometry}
\geometry{top=1cm,left=1cm,right=1cm,bottom=1cm}
\pagestyle{empty}
\usepackage{color,soul}
\newcommand{\hlc}[2]{\sethlcolor{#1} \hl{#2}}
\definecolor{lightyellow}{RGB}{255,255,150}
\definecolor{skyblue}{RGB}{180,255,255}
\newcommand{\gef}[1]{\hlc{skyblue}{#1}}
\newcommand{\gdb}[1]{\hlc{lightyellow}{#1}}

% Redefine section commands to use less space
\makeatletter
\renewcommand{\section}{\@startsection{section}{1}{0mm}%
                                {-1ex plus -.5ex minus -.2ex}%
                                {0.5ex plus .2ex}%x
                                {\normalfont\large\bfseries}}
\renewcommand{\subsection}{\@startsection{subsection}{2}{0mm}%
                                {-1explus -.5ex minus -.2ex}%
                                {0.5ex plus .2ex}%
                                {\normalfont\normalsize\bfseries}}
\renewcommand{\subsubsection}{\@startsection{subsubsection}{3}{0mm}%
                                {-1ex plus -.5ex minus -.2ex}%
                                {1ex plus .2ex}%
                                {\normalfont\small\bfseries}}
\makeatother

\setcounter{secnumdepth}{0} % Don't print section numbers
\setlength{\parindent}{0pt}
\setlength{\parskip}{3pt plus 0.5ex}
\begin{document}
%\raggedright
\footnotesize
\begin{multicols*}{3}
% multicol parameters
% These lengths are set only within the two main columns
%\setlength{\columnseprule}{0.25pt}
\setlength{\premulticols}{1pt}
\setlength{\postmulticols}{1pt}
\setlength{\multicolsep}{1pt}
\setlength{\columnsep}{2pt}

\begin{center}
     \Large{\textbf{Reversing with \gdb{GDB} and \gef{GEF}}} \\
\end{center}

% Gdb commands are \gdb{highlighted in yellow},
% while GEF commands are \gef{highlighted in cyan}.
%
\section{Starting GDB}
\texttt{gdb}\textit{ program }[\textit{core}|\textit{pid}] \\
\texttt{gdb}\textit{ gdb-options }[\texttt{-{}-args}\textit{ program args}] \\
\texttt{gdb -p} \textit{pid} \\

At startup, gdb reads the following init files and executes their commands:
\texttt{/etc/gdb/gdbinit} and \texttt{\textasciitilde/.gdbinit}, plus \texttt{./.gdbinit},
if \texttt{set auto-load local-gdbinit} is set to \texttt{on}.
In addition to reading its config from \texttt{\textasciitilde/.gef.rc}, GEF can also be configured with \gef{\texttt{gef config}}.

Some options are:\\
\begin{tabular}{@{}ll@{}}
\texttt{-q}/\texttt{-{}-quiet}/\texttt{-{}-silent} & don't print version number on startup \\
\texttt{-h}/\texttt{-{}-help} & print help \\
\texttt{-{}-tty=}\textit{TTY} & use \textit{TTY} for I/O by debugged program \\
\texttt{-{}-nh} & do not read \texttt{\textasciitilde/.gdbinit} \\
\texttt{-x}\textit{ FILE} & execute GDB commands from \textit{FILE} \\
\texttt{-ix}\textit{ FILE} & like \texttt{-x} but execute before loading inferior \\
\texttt{-ex}\textit{ CMD} & execute a single GDB command \\ % ; may be \\ & used multiple times and with \texttt{-x} \\
\texttt{-iex}\textit{ CMD} & like \texttt{-ex} but before loading inferior \\
\texttt{-s}\textit{ SYMFILE} & read symbols from \textit{SYMFILE} \\
% \texttt{-{}-readnow} & fully read symbol files on first access \\
% \texttt{-{}-readnever} & do not read symbol files \\
\texttt{-{}-write} & set writing into executable and core files \\
\end{tabular}

To quit, \gdb{\texttt{q}[\texttt{uit}]} or \textit{Ctrl-D}.
You can invoke commands on the standard shell by using:
\gdb{\texttt{shell}\textit{ command-string}}
or simply: \gdb{\texttt{!}\textit{command-string}} \\
Commands can be abbreviated,
when unambiguous, and some commands can be repeated by typing just
\textit{Return}. \textit{Tab} autocompletes.
You can ask for information on commands by using \gdb{\texttt{h}[\texttt{elp}]} and \gef{\texttt{gef help}},
or use \gdb{\texttt{apropos} \textit{regexp}} to search for commands matching \textit{regexp}.

When GDB is spawn from a tmux session, \gef{\texttt{tmux-setup}} splits the pane vertically, and configure the context to be redirected to the new pane.

\paragraph{Debugging targets}
\gdb{\texttt{target} \textit{type param}} connects to machines, processs, or files; e.g.
\texttt{target remote | sshpass -p\textit{pw} ssh -T [\texttt{-p} \textit{port}] [\textit{user}\texttt{@}\textit{host}] gdbserver - \textit{prog} [\textit{args}]}

\gef{\texttt{gef-remote}} is a wrapper for the \texttt{target remote} command,
which includes a QEMU-mode with \texttt{-q}.

\paragraph{Processes and threads}
When a program forks, gdb will continue to debug the parent.
To follow the child, use \gdb{\texttt{set follow-fork-mode}};
to debug both processes, use
\gdb{\texttt{set detach-on-fork off}}. Then, you can list the inferiors with \gdb{\texttt{info inferiors}} and switch among them with \gdb{\texttt{inferior} \textit{id}}.
If you issue a \texttt{run} after an \texttt{exec} call executes, the new target restarts.
To restart the original one, use \gdb{\texttt{file}} with parent executable name as its argument.

\gdb{\texttt{info threads}} lists current threads, \gdb{\texttt{thread} \textit{id}} selects the current one, \gdb{\texttt{thread name} \textit{name}} gives a name and \gdb{\texttt{thread apply} {\textit{range}|\texttt{all}} \textit{cmd}} runs \textit{cmd} for each specified thread; e.g.~\texttt{thread apply all bt}.

In (default) all-stop mode, whenever your program stops, all threads stop. Conversely, whenever you restart the program, all threads start executing, \emph{even when single-stepping}. Set \texttt{scheduler-locking} to \texttt{on} or \texttt{step} to avoid this. To enable non-stop mode, use \gdb{\texttt{set pagination off}} and \gdb{\texttt{set non-stop on}}, \emph{before} starting/connecting-to a target.

\paragraph{Logging output}
Logging can be enabled/disabled with \gdb{\texttt{set logging on}/\texttt{off}}.
\gdb{\texttt{set logging file}\textit{ file}} changes the current logfile
(default: \texttt{gdb.txt}).
\gdb{\texttt{show logging}} shows current logging settings; other settings are
\texttt{logging overwrite}, and
\texttt{logging redirect} to choose whether the output goes to both terminal and logfile.

\section{Starting your program}
\gdb{\texttt{r}[\texttt{un}] [\textit{args}]},
\gdb{\texttt{start} [\textit{args}]} and
\gdb{\texttt{starti} [\textit{args}]}

\texttt{start} does the equivalent of setting a temporary breakpoint
at the beginning of \emph{main} and invoking \texttt{run}.
\texttt{starti} sets a temporary bp at the first instruction and invokes \texttt{run}.
For programs containing an elaboration phase, \texttt{starti} stops execution at the start of such a phase.
% In GEF, see also \gef{\texttt{entry-break}}/\gef{\texttt{start}}.

\textit{args} may include ``*'', or ``[...]''; they are expanded using the
shell that will start the program (specified by the \texttt{\$SHELL} environment
variable).  Input/output redirection with ``\texttt{>}'', ``\texttt{<}'', or ``\texttt{>{}>}'' are also allowed. \\
With no arguments these commands use arguments last specified;
to cancel previous arguments, use \gdb{\texttt{set args}} without arguments.
To start the inferior without using a shell, use \gdb{\texttt{set startup-with-shell off}}.

\gdb{\texttt{set disable-randomization on}} (enabled by default) turns off ASLR;
equivalently: \gdb{\texttt{set exec-wrapper setarch `uname -m` -R}}. Also, \gef{\texttt{aslr}}.

\gef{\texttt{elf}[\texttt{-info}]} provides basic information on currently loaded program.
\gef{\texttt{checksec}}, inspired from \texttt{checksec.sh}, provides a convenient way to determine which security protections are enabled in a binary.

Signal handling/delivering: \gdb{\texttt{info signals}} and \gdb{\texttt{handle} \textit{signal action}}.

\subsection{Environment}
\gdb{\texttt{show environment} [\textit{varname}]}, \gdb{\texttt{set environment varname} [\texttt{=}\textit{value}]} and \gdb{\texttt{unset environment}\textit{ varname}}

The changes are for your program (and the shell gdb uses to launch it), not for gdb itself.
% If your shell runs an initialization file when started non-interactively, it may affect your program.

\gdb{\texttt{set cwd }[\textit{dir}]} sets the inferior's working directory to \textit{dir},
while \\ \gdb{\texttt{cd }[\textit{dir}]} changes gdb working directory.

\gdb{\texttt{tty}} is an alias for \texttt{set inferior-tty}, which can be used
to set the terminal terminal that will be used for future runs.

If debugged program was compiled with \texttt{-fstack-protector}, \gef{canary} displays the value of the canary.
\gef{\texttt{hijack-fd} \textit{fd-num} \textit{newfile}} can be used to modify file descriptors of the debugged process.
Permission rights on specific pages can be set by using \gef{\texttt{set-permission}}.

\section{Stopping execution}
\begin{itemize}
\item \emph{breakpoints} make your program stop whenever a certain point in the program is reached; you can add conditions to control whether your program stops.
\gdb{[\texttt{t}]\texttt{b}[\texttt{reak}] \textit{location} [\texttt{thread} \textit{id}] [\texttt{if} \textit{condition}]} sets a [temporary] breakpoint at the given \textit{location} [, and given \textit{condition}]. When called without any arguments, \gdb{\texttt{break}} sets a breakpoint at the next instruction to be executed. To use an address as a location, you must prepend it with \texttt{*}; e.g. \texttt{*0x1234}.
	\begin{itemize}
	\item \gdb{[\texttt{t}]\texttt{hbreak} \textit{args}} sets [temporary] \emph{hardware} breakpoints
	\item \gdb{\texttt{rbreak} [\textit{file}\texttt{:}]\textit{regex}} set breakpoints on all functions [of \textit{file}] matching the regular expression \textit{regex}.
	\end{itemize}
\gdb{\texttt{info break}[\texttt{points}] [\textit{list}]} prints all breakpoints, watchpoints, and catchpoints (or the ones specified in \textit{list}).
\gdb{\texttt{clear} \textit{location}} and \gdb{\texttt{delete} [\texttt{breakpoints}] \textit{list}} delete breakpoints.

\gdb{\texttt{ignore} \textit{bnum} \textit{n}} sets the ignore count of breakpoint \textit{bnum} to \textit{n} (0 to make it stop the next time breakpoint \textit{bnum} is reached).
\item \gef{\texttt{format-string-helper}} creates a specific type of breakpoints dedicated to detecting potentially insecure format strings. %  when using the GlibC library.
\item \gef{\texttt{pie} \ldots} handles PIE breakpoints.
\item \emph{watchpoints}, AKA \emph{data breakpoints}, are special breakpoints that stop your program when the value of an expression changes.
	\gdb{\texttt{watch} [\texttt{-l}[\texttt{ocation}]] \textit{expr} [\texttt{thread} \textit{thread-id}]} sets a watchpoint for expression \textit{expr}.
If the command includes \textit{thread-id} argument, gdb breaks only
when the corresponding thread changes the value of expr.
Ordinarily a watchpoint respects the scope of variables in \textit{expr}; however,
\texttt{-location} tells gdb to instead watch the memory referred to
by \textit{expr}.
	\begin{itemize}
	\item \gdb{\texttt{rwatch}} sets a \emph{read} watchpoint
	\item \gdb{\texttt{awatch}} sets an \emph{access (i.e., read or write)} watchpoint
	\item \gdb{\texttt{info watchpoints [\textit{list}]}} prints a list of watchpoints
	\end{itemize}
\item \emph{catchpoints} stop your program when a certain kind
of event occurs, such as throwing a C++ exception or loading a library.
\gdb{\texttt{catch} \textit{event}} stops when \textit{event} occurs; it can be:
an exception event;
\texttt{syscall [\textit{name} | \textit{number} | \texttt{g}[\texttt{roup}]:\textit{groupname}]},
\texttt{exec}, \texttt{fork},
[\texttt{un}]\texttt{load} \textit{regexp}, \texttt{signal }[\textit{signal...} | \texttt{all}].
Use \gdb{\texttt{set stop-on-solib-events 1}} to stop the target when a shared library is loaded or unloaded.
\item To stop when your program receives a signal, use \gdb{\texttt{handle}}
\end{itemize}
A *point can have any of several different states of enablement:
\begin{itemize}
\item Enabled/Disabled--- \gdb{\{\texttt{enable}|\texttt{disable}\} [\texttt{breakpoints}] [\textit{list}]}
\item Enabled once --- \gdb{\texttt{enable} [\texttt{breakpoints}] \texttt{once} [\textit{list}]}
\item Enabled for a count --- \gdb{\texttt{enable} [\texttt{breakpoints}] \texttt{count} \textit{n} [\textit{list}]}
\item Enabled for deletion --- \gdb{\texttt{enable} [\texttt{breakpoints}] \texttt{delete} [\textit{list}]}
\end{itemize}
\gdb{\texttt{condition} \textit{bnum} [\textit{expr}]} sets or removes a condition for *point \textit{bnum}.

When a *point is hit, GEF displays the context (current instruction, registers, stack, \ldots) via its
\gef{\texttt{context}} command; to configure its layout see
\gef{\texttt{gef config context.layout}}.
Any *point can execute a series of commands when hit; see \gdb{\texttt{commands}}.

\gdb{\texttt{save breakpoints} \textit{filename}} saves all current *point definitions together with their commands and ignore counts, into \textit{filename}.
To read the saved breakpoint definitions, use \gdb{\texttt{source}}.

\subsection{Checkpoints}
% On Linux gdb can save a snapshot of a program's state, called a \emph{checkpoint}, and come back to it later.  Returning to a checkpoint effectively undoes everything that has happened in the program since the checkpoint was saved.

\gdb{\texttt{checkpoint}} saves a snapshot;
\gdb{\texttt{info checkpoints}} lists the checkpoints that have been saved,
while
\gdb{\texttt{restart }\textit{checkpoint-id}} restores the state that was saved.
% All program variables, registers, stack frames etc. will be returned to the values
% that they had when the checkpoint was saved.
% Note that breakpoints, gdb variables, command history etc. are not affected
% by restoring a checkpoint. In general,
A checkpoint only restores things that
reside in the program being debugged, not in the debugger.
Finally, \gdb{\texttt{delete checkpoint}\textit{ checkpoint-id}} deletes the corresponding
checkpoint.

Returning to a checkpoint restores the user state of the program
being debugged, and a subset of the OS state including file pointers.
% It won't ``un-write'' data from a file, but it will rewind the file pointer to the previous location,
% so that the previously written data can be overwritten. For files opened in read mode, the
% pointer will also be restored so that the previously read data can be read again.

\section{Getting information}
\gdb{\texttt{i}[\texttt{nfo}]} is for describing the state of your program. For
example, you can show the arguments passed to a function with \gdb{\texttt{info args}};
you can get a complete list of the info sub-commands with \texttt{help info}.

\gdb{\texttt{info program}} displays information about the status of your program.
\gef{\texttt{process-status}} provides an exhaustive description of the current running process, by extending the information provided by \gdb{\texttt{info proc}}. %  with information from procfs.
\gef{got} displays the status of the GOT inside the process.

\gef{\texttt{vmmap}} displays the entire memory space mapping ($\!\!$\gdb{\texttt{info shared}} just the status of shared libraries); \gef{\texttt{xfiles}} is a more convenient representation of \gdb{\texttt{info files}} to list the targets and files being debugged.

You can assign the result of an expression to an environment variable with \gdb{\texttt{set}}.
For example, to set the prompt to a \$-sign: \texttt{set prompt \$}.

\gdb{\texttt{show}} is for describing the state of gdb itself. You can
change most of things by using \texttt{set};
for example, what number system is used with \texttt{set
radix}, or inquire it with \texttt{show radix}.
To display all settable parameters and their values, you can use
\texttt{show} with no arguments. % ; you may also use \texttt{info set}: both print the same.

\gdb{\texttt{info address} \textit{symbol}} shows where data for \textit{symbol} is stored.
\gdb{\texttt{info symbol }\textit{addr}} prints
the name of a symbol stored at the address \textit{addr}. If no symbol
is stored exactly at \textit{addr}, gdb prints the nearest symbol and an offset from it.
% This is the opposite of the \texttt{info address} command. You can use it to find out the name of a variable or a function given its address.
\gef{\texttt{xinfo}} displays all information about a specific address.
\gdb{[\texttt{add-}]\texttt{symbol-file}} loads different symbol tables.

\subsection{Examining the stack}
One of the stack frames is selected and many commands refer implicitly to it.
There are commands to select whichever frame you are interested in (\gdb{\texttt{f}[\texttt{rame}] [\textit{n}|\textit{addr}]}, \gdb{\texttt{up} [\textit{n}]} and \gdb{\texttt{down} [\textit{n}]}).
When your program stops, gdb automatically selects the currently executing frame and describes it briefly, similar to the \gdb{\texttt{f}[\texttt{rame}]} command (without argument;
for a more verbose description, \gdb{\texttt{info f[\textit{rame}] [\textit{addr}]}}). \\
\gdb{\texttt{info args}} prints the arguments of selected frame, while \gdb{\texttt{info locals}}
the local variables.

A backtrace is a summary of how your program got where it is.
% It shows one line per frame,
% for many frames, starting with the currently executing frame (frame zero), followed by its
% caller (frame one), and on up the stack.
To print a backtrace of the entire stack, use \gdb{\texttt{bt} / \texttt{backtrace} [\texttt{full}]}; for dumping the stacks of all threads: \gdb{\texttt{thread apply bt} [\texttt{full}]}.

\subsection{Examining code}
\gdb{\texttt{disassemble}}
dumps a range of memory as machine instructions, with
the raw instructions in hex with the \texttt{/r} modifier.
The arguments specify a range of addresses to dump,
in one of two forms: \textit{start}\texttt{,}\textit{end} to disassemble
from \textit{start} (inclusive) to \textit{end} (exclusive), and
\textit{start}\texttt{,+}\textit{length} to disassemble
from \textit{start} (inclusive) to \textit{start+length} (exclusive).
See \gef{\texttt{cs} / \texttt{capstone-disassemble}} (note: different args) for using \emph{Capstone}.

If you have \emph{Keystone}, then GEF can assemble native instructions directly with \gef{\texttt{asm} / \texttt{assembler}}.

\subsection{Examining data}
\gef{\texttt{registers} [$r_1$ [$r_2$ \ldots]]} displays registers and dereference any pointers
(in plain gdb you can see register values with \gdb{\texttt{info registers}}); \gef{[\texttt{edit-}]\texttt{flags} [\{\texttt{+}|\texttt{-}|\texttt{\textasciitilde}\} \textit{flag-name} [\ldots]]}
shows/edits the flag register.

\gdb{\texttt{p}[\texttt{rint}] [\texttt{/}\textit{f}] \textit{expr}} prints the
value of expression \textit{expr}
(in the source language). By default it's printed in a format appropriate to its type;
you can choose a different format by specifying \texttt{/}\textit{f}, where \textit{f} is one of: \texttt{o}(octal), \texttt{x}(hex), \texttt{d}(decimal), \texttt{u}(unsigned decimal), \texttt{t}(binary), \texttt{f}(float), \texttt{a}(address), \texttt{c}(char), \texttt{s}(string) and \texttt{z}(hex, zero padded on the left).

\gdb{\texttt{ptype} [\texttt{/o}] \textit{expr}} prints the type of \textit{expr} [with field offsets/sizes].
The expression \texttt{\{\textit{t}\}}\textit{addr} refers to an object of type \textit{t}, stored at  \textit{addr}.

\texttt{print}ed expressions may include function calls in the program being debugged;
use \gdb{\texttt{call} \textit{addr}} for calling functions without printing.

\gdb{\texttt{x}[\texttt{/}\textit{fmt}] [\textit{addr}]} examines memory;
default \textit{addr} is usually just after the last address examined.
\textit{fmt} is a repeat count followed by a format letter and a size letter.
Format letters are like \texttt{print} plus \texttt{i}(instruction).
Sizes are \texttt{b}(byte), \texttt{h}(halfword), \texttt{w}(word), \texttt{g}(giant, 8 bytes).
If a negative number is specified, memory is examined backward. With GEF you can use
\gef{\texttt{hexdump} \{\texttt{qword}|\texttt{dword}|\texttt{word}|\texttt{byte}\} \textit{location} [\texttt{l}\textit{size}] [\texttt{up}|\texttt{down}]}.

\gdb{\texttt{dprintf} \textit{location}\texttt{,}\textit{format-str}[\texttt{,}$a_1$[\texttt{,}$a_2$\ldots]]} sets a dynamic printf at specified \textit{location}; this is actually a dprintf-breakpoint and you can use \gdb{\texttt{cond}} to add a condition, for instance.

\gef{\texttt{memory} \ldots} adds or removes address ranges to the memory view.

\gef{\texttt{dereference} / \texttt{telescope}} simplifies the dereferencing of an address in GDB to determine the content it actually points to.

The \gef{\texttt{\$}} command mimics WinDBG \texttt{?} command:
when provided one argument, it evaluates the expression and displays the result in various formats.
With two arguments, the delta between them.

\gef{\texttt{heap} \ldots} is a base command to get information about the Glibc heap structure; moreover,
\gef{\texttt{heap-analysis-helper}} aims to help the process of identifying Glibc heap inconsistencies by tracking and analyzing allocations and deallocations.

\gef{\texttt{nop}} and \gef{\texttt{patch}} writes values into memory.

\gdb{\texttt{dump}}, \gdb{\texttt{append}}, and \gdb{\texttt{restore}} copy data between target memory and a file. \texttt{dump} and \texttt{append} write data to a file, whereas \texttt{restore} reads data from a file back into the inferior's memory.

\gef{\texttt{print-format} / \texttt{pf}} dumps an arbitrary location as an array of bytes following the syntax of the programming language specified.

\gef{\texttt{shellcode}} is a client for \texttt{@JonathanSalwan} shellcodes database.

\gef{\texttt{stub}} stubs out functions, optionally specifying the return value.

\paragraph{Automatic display}
If you find that you want to print the value of an expression frequently
you might want to add it to the automatic display list so that gdb prints its
value each time your program stops. \\
\gdb{\texttt{display}[\texttt{/}\textit{fmt}] \{\textit{expr}|\textit{addr}\}}
adds the expression/address to such a list.

\subsection{Searching in memory}
\gdb{\texttt{find} [\texttt{/}\textit{size-char}] [\texttt{/}\textit{max-count}] \textit{start-address}\texttt{,} \{\textit{end-address}|\texttt{+}\textit{length}\}, \textit{expr1} [\texttt{,} \textit{expr2 ...}]} where
\textit{size-char} is one of b,h,w,g for 8,16,32,64 bit values respectively,
and if not specified the size is taken from the type of the expression
in the current language;
in the case of C-like languages
a search for an untyped 0x42 will search for "(int) 0x42"
which is typically four bytes, and a search for a string "hello" will
include the trailing \texttt{'\textbackslash 0'}.  The null terminator can be removed from
searching by using casts, e.g.: {char[5]}"hello".

The address of the last match is stored as the value of \texttt{\$\_}.
Convenience variable \texttt{\$numfound} is set to the number of matches.

\gef{\texttt{search-pattern}}, alias \gef{\texttt{grep}}, allows you to search for a specific pattern at runtime in all the segments of your process memory layout.

\gef{\texttt{pattern} \ldots} creates or searches a De Bruijn cyclic pattern to facilitate determining offsets in memory; the algorithm is the same as the one in \emph{pwntools}, and can therefore be used in conjunction.

\gef{\texttt{scan} \textit{haystack} \textit{needle}} searches for addresses that are located in memory mapping \textit{haystack} that belonging to another, \textit{needle};
its arguments are grepped against the processes memory mappings (just like \texttt{vmmap} to determine the memory ranges to search).

\section{Resuming execution}
\gdb{\texttt{c}[\texttt{ontinue}] [\textit{ignore-count}]}
resumes program execution; the optional argument
\textit{ignore-count} allows you to specify a further number of times to ignore a breakpoint at this location. % (if the execution has stopped due to a breakpoint; otherwise the argument is ignored).

To resume execution at a different place, use \gdb{\texttt{return} [\textit{expr}]} to go back to the
calling function; or \gdb{\texttt{j}[\texttt{ump}]} to go to an arbitrary location.
Note that \texttt{return} does not resume execution: it leaves the program stopped in the
state that would exist if the function had just returned. In contrast,
\gdb{\texttt{fin}[\texttt{ish}]} continues running until just after function in the selected stack frame returns.
\gdb{\texttt{u}[\texttt{ntil}]}, without arguments, continues running until a source line past the current line is reached. It is like \texttt{next}, except that when \texttt{until} encounters a jump, it automatically continues execution until the program counter is greater than the address of the jump. \gdb{\texttt{u}[\texttt{ntil}] \textit{location}} continue running until either the specified \textit{location} is reached, or the current stack frame returns.

\subsection{Stepping}
\gdb{\texttt{s}[\texttt{tep}] [\textit{count}]} continue running your program until control reaches a different source line [\textit{count} times]. \gdb{\texttt{n}[\texttt{ext}] [\textit{count}]} continues to the next source line in the current stack frame.
\gdb{\texttt{stepi}/\texttt{si} \textit{count}} and \gdb{\texttt{nexti}/\texttt{ni} \textit{count}} work on machine instructions, instead of source lines.

\subsection{Emulation}
If you have installed \emph{Unicorn} emulation engine and its Python bindings,
\gef{\texttt{unicorn-emulate} / \texttt{emu}} replicates the current memory mapping, and by default (i.e. without any additional argument), it will emulate the execution of the instruction about to be executed (i.e. the one pointed by \texttt{\$pc}) and display which register(s) is(are) tainted by it.

\section{Record, replay and reverse execution}
\gdb{\texttt{record}} starts the process record and replay target;
\gdb{\texttt{record stop}} stops.
% When you stop in \emph{record mode} the inferior process will be left in the same state as if the recording never happened.  In \emph{replay mode}, the inferior process will become ``live'' at that earlier state, and it will then be possible to continue the usual live debugging.

\gdb{\texttt{record goto \{\texttt{begin}|\texttt{end}|\textit{n}\}}} goes to \{begin, end, instruction number~\textit{n} in instruction log\}.

\gdb{\texttt{rc} / \texttt{reverse-continue} \textit{count}} starts executing \emph{in reverse}; similarly, for \gdb{\texttt{reverse-}\{\texttt{step}[\texttt{i}],\texttt{next}[\texttt{i}]\} [\textit{count}]}.
\gdb{\texttt{reverse-finish}} returns to a point where current function was called.

\gdb{\texttt{set exec-direction \{reverse,forward\}}}
affects commands \texttt{step}[\texttt{i}], \texttt{next}[\texttt{i}], \texttt{continue}
and \texttt{finish}; note \texttt{return} cannot be used in reverse.

\rule{1.0\linewidth}{0.25pt}
\scriptsize
Copyright \copyright 2018--2019 by zxgio; cheat-sheet built on \today

This cheat-sheet may be freely distributed under the terms of the GNU
General Public License; the latest version can be found at: \\
\url{https://github.com/zxgio/gdb_gef-cheatsheet}
\end{multicols*}
\end{document}
